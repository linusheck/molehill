\documentclass[tikz]{standalone}
\usepackage[
    group-digits=integer, group-minimum-digits=4, % group digits by thousands
    free-standing-units, unit-optional-argument, % easier input of numbers with units
    ]{siunitx}[=v2]
\usepackage{pgfplots}
\pgfplotsset{
    compat=1.18,
    log ticks with fixed point, % no scientific notation in plots
    table/col sep=tab, % only tabs are column separators
    unbounded coords=jump, % better have skips in a plot than appear to be interpolating
    filter discard warning=false, % Don't complain about empty cells
    width=12cm,
    height=5cm
    }

\usepackage{xcolor}
\usepgfplotslibrary{colorbrewer}
\pgfplotsset{cycle list/Set1}

\SendSettingsToPgf % use siunitx formatting settings in PGF, too

\newcommand{\pathtostuff}{}

\begin{document}

\begin{tikzpicture}
\begin{semilogyaxis}[
    % Which column to be taken from each file
        /pgfplots/table/header=false,
        % axis labels
        xlabel=solved instances,
        ylabel=CPU time (\second),
        % axis ranges
        xmin=0,
        ymin=0.2,
        ymax=6000,
        mark repeat=500,
        % cycle list name=mycolors,
        cycle multiindex* list={
            Set1 \nextlist
            mark list
        },
        mark repeat=10,    % Show a marker every 10 data points
        mark size=2pt,     % Adjust the marker size as you wish
        % legend
        % legend font size smaller
        legend style={nodes={scale=0.5, transform shape}},
        legend entries={PAYNT-CEGIS,SMPMC,SMT(LRA),},
        every axis legend/.append style={at={(1,0)}, anchor=south east, outer xsep=5pt, outer ysep=5pt,},
        ]
        \foreach \tool in { comparison_trees.2025-07-29_15-15-10.results.PAYNT-CEGIS.Benchmarks.xml.bz2.quantile.csv,comparison_trees.2025-07-29_15-15-10.results.SMPMC.Benchmarks.xml.bz2.quantile.csv,comparison_trees.2025-07-29_15-15-10.results.SMT(LRA).Benchmarks.xml.bz2.quantile.csv} {
            \addplot+ [line width=1.5pt] table[y index=5] {\pathtostuff\tool};
        }
\end{semilogyaxis}
\end{tikzpicture}

\end{document}
